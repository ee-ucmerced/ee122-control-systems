\makeatletter
\def\input@path{{../styles/}{../../styles/}{../../../styles/}{../}{../../}{../../../}}
\makeatother


\documentclass{ee122_pset}

% Assignment info
\author{\rule{3cm}{0.4pt}} % Name placeholder
\submitdate{\rule{3cm}{0.4pt}} % Submission date placeholder
\problemset{Problem Set \#4: Your first controller}
\renewcommand{\duedate}{February 18, 2026}
\shorttitle{Problem Set \#4}

\begin{document}

\problem{1}
The Chapter 4 of the FBS textbook (2nd Edition) is on various system examples and the kinds of control problems that arise in those systems. The chapter discusses the following examples: 
    \begin{enumerate}
    \item Cruise Control
    \item Bicycle Dynamics
    \item Operational Amplifier Circuits
    \item Computing Systems and Networks
    \item Atomic Force Microscopy
    \item Drug Administration
    \item Population Dynamics
    \end{enumerate}
Your task in this problem is to thoroughly read ONE of the examples above and apply the techniques you have learned in this course so far to describe the system mathematically, analyze the system properties, and discuss at least one control problem (mathematically) for that system. 

\problempart Choose one of the examples above and do the following:
\begin{itemize}
    \item \textbf{[10 points]} Describe the system mathematically by writing down the governing equations (ODEs, PDEs, TF, FSM etc.) that describe the system dynamics. You can use the equations in the textbook as a starting point. Once you have the model, describe whether the model is linear or nonlinear, time-invariant or time-varying, and SISO or MIMO (single-input single-output or multi-input multi-output).
    \item \textbf{[20 points]} Find at least two properties / metrics that are relevant for the system you chose. For example, if you choose the cruise control example, you can find the rise time and the overshoot of the step response. 
    \item \textbf{[20 points]} Discuss at least one control problem for the system you chose. You may need to read external material to identify control problem for the system of your choice. No need to solve the problem fully --- a description with at least 1-2 accompanying equations about the control problem is sufficient. 
\end{itemize}

\vspace*{\fill}
\begin{center}
[turn to next page]
\end{center}
\problem{2}
The goal of this problem is to design two controllers for a thermostat system: a ``bang-bang'' controller and a proportional controller for a thermostat system. You already know what the proportional controller is from the lectures. Here is some information about the bang-bang, also called, the on-off control: An on-off controller is a simple controller that applies maximum control effort when the system is below a certain threshold and turns off the control effort when the system is above that threshold. Consider a simple model of a thermostat system where the temperature \( T(t) \) of a room is governed by the following first-order ODE:
\[
\frac{dT(t)}{dt} = -\alpha (T(t) - T_{\text{a}}) + \beta u(t)
\]
where \( \alpha > 0 \) is the cooling coefficient, \( T_{\text{a}} \) is the ambient temperature, \( \beta > 0 \) is the heating coefficient, and \( u(t) \) is the control input. Your task is to design two controllers: a proportional controller and a bang-bang controller that maintains the room temperature at a desired setpoint \( T_{\text{set}} \) --- this is a unit step input. 

With the proportional controller, the control input \( u(t) \) is given by:
\[
u(t) = K_p (T_{\text{set}} - T(t))
\]
where \( K_p \) is the proportional gain. 

With the on-off controller, The controller should turn the heater on when the temperature is below \( T_{\text{set}} \) and turn it off when the temperature is above \( T_{\text{set}} \). When you turn the heater on, the control input \( u(t) \) should be set to a maximum value \( u_{\text{max}} \), and when you turn it off, \( u(t) \) should be set to zero. That is, 
\[
u(t) = \begin{cases}
u_{\text{max}}, & \text{if } T(t) < T_{\text{set}} \\
0, & \text{if } T(t) \geq T_{\text{set}}
\end{cases}
\]

\problempart \textbf{[5 points]} Draw a block diagram of the closed-loop system with the controller. Label the input, output, and the controller clearly.

\problempart \textbf{[5 points]} Prove that the transfer function of the system without the controller (i.e., the open-loop transfer function from \( u(t) \) to \( T(t) \)), assuming zero initial condition $T(0) = 0$ is 
\[
G(s) = \frac{\beta}{s + \alpha}.
\]
\problempart \textbf{[10 points]} For the proportional controller, derive the closed-loop transfer function from the setpoint \( T_{\text{set}} \) to the temperature \( T(t) \).
\problempart \textbf{[5 points]} Is the on-off controller linear or nonlinear? 
\problempart \textbf{[20 points]} Simulate the closed-loop system with both controllers: proportional and the on-off controller for a given set of parameters \( \alpha, \beta, T_{\text{a}}, T_{\text{set}} \) and initial temperature \( T(0) \). Plot the temperature response \( T(t) \) over time and discuss the behavior of the system under your controller. You may use MATLAB or Python for the simulation. You must include your code and the resulting plot with your submission.
\problempart \textbf{[5 points]} Discuss the advantages and disadvantages of using each controller. Based on your research, find out if real-world thermostats use proportional control or on-off control?

\end{document}