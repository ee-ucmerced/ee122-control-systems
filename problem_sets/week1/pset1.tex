\makeatletter
\def\input@path{{../styles/}{../../styles/}{../../../styles/}{../}{../../}{../../../}}
\makeatother


\documentclass{ee122_pset}

% Assignment info
\author{\rule{3cm}{0.4pt}} % Name placeholder
\submitdate{\rule{3cm}{0.4pt}} % Submission date placeholder
\problemset{Problem Set \#1: Introduction to Control Systems}
\renewcommand{\duedate}{January 28, 2025}
\shorttitle{Problem Set \#1}

\begin{document}

% Problem 1
\problem{1}
\textbf{[25 points]} To understand and reflect on control systems, solve Exercise 1.1 in FBS textbook. You must do your own original work in coming up with feedback systems. Collaborating to discuss ideas is generally OK, but for this problem, you should spend time on your own thinking about feedback systems first and then you may discuss your answers with others. 


\vspace*{\fill}
\begin{center}
[turn to next page]
\end{center}
\problem{2}
\textbf{[25 points]} To get a feel of control systems and what control actions can achieve, solve Exercise 1.4 in FBS textbook. You may work on MATLAB or Python --- the code for this problem is provided on GitHub: \texttt{cruise\_control.ipynb} OR \texttt{cruise\_control.m}. For Python, you will need the \texttt{python-control} package and for MATLAB, you will need the Control System Toolbox (usually built-in with MATLAB). 

Note: Include your code and plots in your submission. You can use a Jupyter notebook and print it to get a PDF, or include your code and graphs in a document that you can save as PDF. DO NOT use screenshots! Save your graph as images and include them in your document --- another step towards presenting your work professionally. 

\vspace*{\fill}
\begin{center}
[turn to next page]
\end{center}

\problem{3}
Recap EE 102 and answer the following questions:
\problempart \textbf{[5 points]} Describe (in words) a dynamical system of your choice (do not choose constant systems).
\problempart \textbf{[5 points]} Draw a block diagram of the system you described in part (a). Label the inputs and outputs clearly.
\problempart \textbf{[15 points]} Write an ordinary differential equation (should be at least first-order) that describes the system. Note the input and output variables clearly.
\problempart \textbf{[25 points]} By computing the Fourier transform of the ODE, derive the frequency response of the system (transfer function). Show your work clearly.


\textbf{Hint:} Transform your ODE to the frequency domain by applying the Fourier transform to each term in the ODE. Then, recall that the transfer function is defined as the ratio of the output to the input in the frequency domain.

\end{document}