
\subsection{What proportional control is good for (and limits)}
\begin{itemize}
\item Increasing $K_p$ typically increases loop gain, which can move closed-loop poles left (faster decay) \emph{or} can reduce damping (more overshoot), depending on plant structure.
\item For step tracking in unity feedback, proportional control reduces steady-state error for type-0 plants, but does not generally eliminate it:
\[
e_{ss}=\lim_{t\to\infty}e(t)=\lim_{s\to 0} sE(s)=\frac{1}{1+K_p G(0)}.
\]
\end{itemize}

\subsection{General second-order example with $K_p$}

Let the plant be
\[
G(s)=\frac{b_0}{s^2+a_1 s + a_0},
\]
with unity feedback and proportional control $K_p$.
Then
\[
T(s)=\frac{K_p b_0}{s^2+a_1 s + a_0 + K_p b_0}.
\]
So the closed-loop characteristic equation is
\[
s^2+a_1 s + (a_0+K_p b_0)=0.
\]
Match to the standard form $s^2+2\zeta\omega_n s + \omega_n^2$:
\[
2\zeta\omega_n = a_1,
\qquad
\omega_n^2 = a_0+K_p b_0,
\qquad
\Rightarrow\quad
\zeta=\frac{a_1}{2\sqrt{a_0+K_p b_0}}.
\]
\paragraph{Takeaway (important trade-off in this structure).}
Increasing $K_p$ increases $\omega_n$ (often faster), but \emph{decreases} $\zeta$ here:
\[
\omega_n\uparrow \ \Rightarrow\  \zeta=\frac{a_1}{2\omega_n}\downarrow,
\]
so the response can become faster yet more underdamped (more overshoot) as $K_p$ increases.

\paragraph{Steady-state error to a step.}
For unity feedback,
\[
e_{ss}=\frac{1}{1+K_p G(0)}
=\frac{1}{1+K_p\left(\frac{b_0}{a_0}\right)}.
\]

% ------------------------------------------------------------
% Numerical examples
% ------------------------------------------------------------

\section{Numerical examples (proportional control on a second-order plant)}

\subsection{Example A: Overdamped plant with high $e_{ss}$ (type-0, low DC gain)}

Choose
\[
G(s)=\frac{0.1}{s^2+20s+1}.
\]
\paragraph{Open-loop poles (plant poles).}
Solve $s^2+20s+1=0$:
\[
s_{1,2}\approx -19.9499,\ -0.0501
\]
(two real negative poles $\Rightarrow$ overdamped, with a very slow dominant pole at $-0.0501$).

\paragraph{Steady-state error with proportional control.}
Here $G(0)=\frac{0.1}{1}=0.1$, so
\[
e_{ss}=\frac{1}{1+0.1K_p}.
\]
If $K_p=1$:
\[
e_{ss}=\frac{1}{1+0.1}=0.909\quad (\text{very large}).
\]

\paragraph{Closed-loop transfer function and poles.}
With unity feedback:
\[
T(s)=\frac{0.1K_p}{s^2+20s+(1+0.1K_p)}.
\]
Closed-loop poles are the roots of
\[
s^2+20s+(1+0.1K_p)=0.
\]

\paragraph{Two $K_p$ values (show the pole movement).}
\begin{itemize}
\item If $K_p=1$, characteristic equation is $s^2+20s+1.1=0$:
\[
s_{1,2}\approx -19.9448,\ -0.0552
\]
(still overdamped; dominant pole slightly faster than $-0.0501$).
\item If $K_p=50$, characteristic equation is $s^2+20s+6=0$:
\[
s_{1,2}\approx -19.6954,\ -0.3046
\]
(still overdamped; dominant pole moves left a lot $\Rightarrow$ much faster settling).
\end{itemize}

\paragraph{Corresponding steady-state errors.}
\[
K_p=1:\ e_{ss}=0.909,
\qquad
K_p=50:\ e_{ss}=\frac{1}{1+5}=0.166.
\]
\paragraph{Lesson.}
This example cleanly shows: increasing $K_p$ can (i) reduce $e_{ss}$ (by increasing low-frequency loop gain) and (ii) move the dominant pole left (faster response), while remaining overdamped over a wide range of $K_p$.

\subsection{Example B: Initially well-damped plant where larger $K_p$ speeds up but reduces damping (more overshoot)}

Choose
\[
G(s)=\frac{1}{s^2+2s+1}=\frac{1}{(s+1)^2}.
\]
\paragraph{Open-loop poles.}
\[
s=-1\ \text{(double pole)} \quad \Rightarrow\quad \text{critically damped plant dynamics.}
\]
\paragraph{Closed-loop with proportional control.}
\[
T(s)=\frac{K_p}{s^2+2s+(1+K_p)}.
\]
Match $s^2+2\zeta\omega_n s + \omega_n^2$:
\[
2\zeta\omega_n=2,\qquad \omega_n^2=1+K_p
\quad\Rightarrow\quad
\omega_n=\sqrt{1+K_p},\ \ \zeta=\frac{1}{\sqrt{1+K_p}}.
\]
So $K_p\uparrow$ makes $\omega_n\uparrow$ (faster), but $\zeta\downarrow$ (less damped).

\paragraph{Pick two $K_p$ values.}
\begin{itemize}
\item If $K_p=0$, then $\omega_n=1$ and $\zeta=1$ (critically damped).
\item If $K_p=8$, then $\omega_n=\sqrt{9}=3$ and $\zeta=\frac{1}{3}$ (underdamped).
\end{itemize}
Poles for $K_p=8$:
\[
s=-\zeta\omega_n \pm j\omega_n\sqrt{1-\zeta^2}
= -1 \pm j\sqrt{8}.
\]
\paragraph{Steady-state error.}
Here $G(0)=1$, so
\[
e_{ss}=\frac{1}{1+K_p}.
\]
For $K_p=8$, $e_{ss}=\frac{1}{9}\approx 0.111$.

\paragraph{Lesson.}
This example cleanly shows the common trade-off in proportional control for some second-order plants:
\[
K_p\uparrow \Rightarrow \omega_n\uparrow\ (\text{faster}),\quad \zeta\downarrow\ (\text{more overshoot/oscillation}),
\]
while $e_{ss}$ improves because $G(0)$ is large and proportional gain increases low-frequency loop gain.
