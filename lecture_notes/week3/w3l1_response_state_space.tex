\makeatletter
\def\input@path{{../styles/}{../../styles/}{../../../styles/}{../}{../../}{../../../}}
\makeatother
\documentclass{ee122_notes}
% macros.tex - Course meta information
\renewcommand{\course}{EE 122} % with a space
\renewcommand{\coursetitle}{Introduction to Control Systems}
\renewcommand{\instructor}{Ayush Pandey}
\renewcommand{\student}{Name: }

\renewcommand{\semester}{Spring 2026}
\date{\semester} % this sets the LaTeX date field safely

% Problem set number
\renewcommand{\psetnum}{1}

% Release / due — use renewcommand because package provides empties
\renewcommand{\releasedate}{January 20, 2026}
\renewcommand{\duedate}{May 14, 2026}

% The following packages can be found on http:\\www.ctan.org
% \usepackage{graphics} % for pdf, bitmapped graphics files
%\usepackage{epsfig} % for postscript graphics files
%\usepackage{mathptmx} % assumes new font selection scheme installed
%\usepackage{times} % assumes new font selection scheme installed
\usepackage{amsmath} % assumes amsmath package installed
\usepackage{amssymb,mathtools}  % assumes amsmath package installed
\usepackage{xcolor}
\usepackage{pgfplots,subcaption}
\usepackage[hidelinks]{hyperref}
\usepackage{verbatim}
\usepackage{graphicx}
\usepackage{listings}
\usepackage{fancyhdr}
% \usepackage{geometry}
\usepackage{siunitx}
\usepackage[most]{tcolorbox}
\usepackage{enumitem}
\usepackage{environ}
\usepackage{pifont}
% -------- listings (Python) ----------
\lstdefinestyle{py}{
  language=Python,
  basicstyle=\ttfamily\small,
  keywordstyle=\color{blue!60!black}\bfseries,
  commentstyle=\color{green!40!black},
  stringstyle=\color{orange!60!black},
  showstringspaces=false,
  columns=fullflexible,
  frame=single,
  framerule=0.3pt,
  numbers=left,
  numberstyle=\tiny,
  xleftmargin=1em,
  tabsize=2,
  breaklines=true,
}

\usepackage[american]{circuitikz}
\usepackage{tikz}
\usetikzlibrary{arrows.meta,positioning,calc,angles,quotes}
\tikzset{
  >={Latex[length=2.2mm]},
  block/.style={draw, thick, rectangle, minimum height=10mm, minimum width=24mm, align=center},
  gain/.style={block, minimum width=14mm},
  sum/.style={draw, thick, circle, inner sep=0pt, minimum size=6mm},
  conn/.style={-Latex, thick},
}
\usepackage{caption}    
\usepackage{lscape}
\usepackage{soul}
\usepackage{physics}
\usepackage{hyperref}
\hypersetup{
    colorlinks=true,
    linkcolor=blue,
    filecolor=magenta,      
    urlcolor=blue,
    pdftitle={week1_notes},
    pdfpagemode=FullScreen,
}
%\usepackage{float} 

%\usepackage[demo]{graphicx}
\pgfplotsset{compat=1.18}
% \usepgfplotslibrary{fillbetween}

\newsavebox{\measurebox}

\let\proof\relax\let\endproof\relax


\def\abs#1{\left\lvert#1\right\rvert}
\let\proof\relax
\let\endproof\relax
\usepackage{amsthm}
\usepackage{accents}
\usepackage{relsize}
\newcommand{\ubar}[1]{\underaccent{\bar}{#1}}
\newtheorem{theorem}{Theorem}
\newtheorem{corollary}{Corollary}[theorem]
\newtheorem{lemma}{Lemma}
\newtheorem{proposition}{Proposition}
\newtheorem{statement}{Statement}

\theoremstyle{definition}
\newtheorem{definition}{Definition}
 
\theoremstyle{remark}
\newtheorem*{remark}{Remark}
\theoremstyle{remark}
\newtheorem*{claim}{Claim}
\setlength{\parindent}{0cm}
\newenvironment{nalign}{
    \begin{equation}
    \begin{aligned}
}{
    \end{aligned}
    \end{equation}
    \ignorespacesafterend
} 

\renewcommand{\releasedate}{January 28, 2026}

% Simple axis helpers
\newcommand{\stepaxes}{%
\begin{tikzpicture}[x=0.55cm,y=0.75cm,baseline]
  % axes: first quadrant
  \draw[-{Stealth[length=2mm]}] (0,0) -- (7,0) node[below] {$t$};
  \draw[-{Stealth[length=2mm]}] (0,0) -- (0,4) node[left] {$y(t)$};
  % light ticks (optional)
  \foreach \x in {1,2,3,4,5,6} \draw (\x,0) -- (\x,0.12);
  \foreach \y in {1,2,3} \draw (0,\y) -- (0.12,\y);
\end{tikzpicture}%
}

\newcommand{\impaxes}{%
\begin{tikzpicture}[x=0.55cm,y=0.75cm,baseline]
  % axes: first and fourth quadrants (t >= 0, y can be +/-)
  \draw[-{Stealth[length=2mm]}] (0,0) -- (7,0) node[below] {$t$};
  \draw[-{Stealth[length=2mm]}] (0,-3) -- (0,3) node[left] {$y(t)$};
  % light ticks (optional)
  \foreach \x in {1,2,3,4,5,6} \draw (\x,0) -- (\x,0.12);
  \foreach \y in {-2,-1,1,2} \draw (0,\y) -- (0.12,\y);
\end{tikzpicture}%
}



\begin{document}

% (No header here — assumes ee122_notes style is already set up in the main file.)

\section*{Week 3, Lecture 1: Response of Systems Using State-Space}
\subsection*{Instructor: \instructor}
\subsection*{Date: \releasedate}

\subsection*{Learning goals}
By the end of this lecture, you should be able to:
\begin{itemize}
  \item explain how ODE, transfer function, and state-space models relate to each other,
  \item write a state-space model for linear systems and recognize that it is not unique,
  \item state and validate the dimensions of $A,B,C,D$ matrices in state-space models.
\end{itemize}

\subsection*{Recap}
For the past two weeks, we have focused on two main ideas:
\begin{enumerate}
  \item Week 1: Examples of control systems in real life and how we can draw block diagrams to show these systems.
  \item Week 2: Modeling of dynamical systems using ordinary differential equations and transfer functions.
\end{enumerate}

\section{Models of dynamical systems --- three perspectives}
A foundational way to model physical systems is with an ordinary differential equation (ODE) in the time domain.
This is the description that mathematicians use and prefer: it directly encodes how derivatives of variables relate to various inputs and parameters and allows for maximum flexibility in modeling.

For engineers who are designing systems, we often want representations that can quickly provide insight into system behavior and presents the information about what can be changed to get a desired behavior. Transfer functions (TF) descriptions provide this. With TFs, we are able to quickly read off properties in the frequency domain such as poles, stability, time constants, bandwidth --- properties that we are often interested in optimizing. The transfer function is not a new physics model, it is just a convenient
representation obtained by transforming the ODE into the $s$-domain using Laplace transform.

Finally, state-space (SS) is essentially an ODE rewritten in a particular structured way. This is essentially ``dressing up'' the ODE in a special way that is useful to us. It is still time-domain dynamics but organized into first-order vector form so that the linearity of the systems is obvious from the system model. Recall how when you think of something that is linear, you think of ``$y = mx + b$'' style equation. That is precisely the idea with state space models. We write ODEs in a form that looks like ``$\dot{x} = A x + B u$'' which is linear in $x$ and $u$ and collects all the terms that are derivatives on the left-hand side.

Between these three perspectives of modeling a system, a useful mental map is:
\[
\text{ODE (time domain)}
\;\;\longleftrightarrow\;\;
\text{TF (frequency domain)}
\]
and 
\[
\text{ODE (time domain)}
\;\;\longleftrightarrow\;\;
\text{SS (time domain)}.
\]

\subsection{Example: A first-order system}
Consider the first-order linear system
\[
\frac{dy}{dt} + k\,y = u.
\]
This is an ODE model in the time domain. Taking the Laplace transform (assuming zero initial conditions), we get
\[
sY(s) + kY(s) = U(s)
\quad\Rightarrow\quad
\frac{Y(s)}{U(s)} = \frac{1}{s+k}.
\]
This is the transfer function representation in the frequency domain. Finally, if you define a state such as
\[
x = y.
\]
Then
\[
\dot{x} = \dot{y} = -k y + u = -k x + 1\cdot u,
\]
so one valid state-space model is
\[
\dot{x} = A x + B u,
\qquad
A = [-k],
\qquad
B = [1].
\]

However, state-space representation is not unique! Different state choices can describe the same input-output behavior and lead to different matrices of the system. For example, define a new state
\[
x = 2y.
\]
Differentiate:
\[
\dot{x} = 2\dot{y}.
\]
From the ODE, $\dot{y} = -k y + u$. Multiply by $2$:
\[
\dot{x} = 2(-k y + u) = -2k y + 2u.
\]
Now substitute $y = x/2$:
\[
\dot{x} = -2k\left(\frac{x}{2}\right) + 2u = -k x + 2u.
\]
So another valid state-space model is
\[
\dot{x} = A x + B u,
\qquad
A = [-k],
\qquad
B = [2].
\]
This is the same physical system but with a different state definition we end up with a different B matrix, showing us that the state space models are not unique. Let us formally build the definition of state-space models.

\section{State-space representation of linear systems}
\subsection{States as a vector}
The first key idea in developing a state space model is the following:
\begin{quote}
If a system is $n$th order, then we need $n$ states to describe the system.
\end{quote}

Let us call these states $x_1, x_2, \ldots, x_n$. Each state is a variable that describes some aspect of the system. For example, in a mechanical system, states can be positions and velocities of various masses. In an electrical system, states can be voltages across capacitors and currents through inductors.

We collect the states into a vector
\[
\textbf{x} \;=\;
\begin{bmatrix}
x_1\\
x_2\\
\vdots\\
x_n
\end{bmatrix}
\in \mathbb{R}^{n\times 1}.
\]
This is a column vector or a column matrix with $n$ rows and $1$ column.

\subsection{State equation that describes the dynamics}
For a linear system, the dynamics can be written as the linear differential equation
\[
\dot{\textbf{x}} = A \textbf{x} + B u.
\]

In this lecture we focused on the single-input case where the input is a scalar:
\[
u \in \mathbb{R}
\]
however, in general, the input can be a vector as well when you have multiple inputs that control a system.

\begin{popquiz}
What are the sizes of $A$ and $B$ matrices in
\[
\dot{\textbf{x}} = A \textbf{x} + B u,
\qquad
\textbf{x} \in \mathbb{R}^{n\times 1},
\qquad
u \in \mathbb{R}?
\]
\popqsplit
Since $\dot{\textbf{x}}\in\mathbb{R}^{n\times 1}$ and $\textbf{x}\in\mathbb{R}^{n\times 1}$, we need
\[
A \in \mathbb{R}^{n\times n}.
\]
Also, since $u$ is scalar, $Bu$ must land in $\mathbb{R}^{n\times 1}$, so
\[
B \in \mathbb{R}^{n\times 1}.
\]
\end{popquiz}

\subsection{Outputs and the output equation}
If the system has $m$ outputs $y_1, y_2, \ldots, y_m$, we collect them into a column vector
\[
\textbf{y} \;=\;
\begin{bmatrix}
y_1\\
y_2\\
\vdots\\
y_m
\end{bmatrix}
\in \mathbb{R}^{m\times 1}.
\]

To complete the system description, we write the following linear output equation:
\[
\textbf{y} = C\textbf{x} +  D u.
\]

\begin{popquiz}
What are the sizes of $C$ and $D$ matrices in
\[
\textbf{y} = C\textbf{x} +  D u,
\qquad
\textbf{x} \in \mathbb{R}^{n\times 1},
\qquad
\textbf{y} \in \mathbb{R}^{m\times 1},
\qquad
u \in \mathbb{R}?
\]
\popqsplit
We need $C\textbf{x} \in \mathbb{R}^{m\times 1}$ with $\textbf{x}\in\mathbb{R}^{n\times 1}$, so
\[
C \in \mathbb{R}^{m\times n}.
\]
Also $Du$ must be in $\mathbb{R}^{m\times 1}$ with scalar $u$, so
\[
D \in \mathbb{R}^{m\times 1}.
\]
\end{popquiz}
\section{Example: A second-order system }
To highlight why state-space is a useful model ``language,'' we revisit a standard second-order system from last lecture. Consider a spring-mass-damper with input force $F(t)$ and output position $q(t)$ (review last week's notes if needed, we will not repeat the derivation here). The ODE model is
\[
m\ddot{q} + b\dot{q} + k q = F(t)
\]
where $q$ is the position of the mass, $m$ is the mass, $b$ is the damping coefficient, and $k$ is the spring constant.
Using Laplace transform with zero initial conditions, the transfer function from input $F$ to output $q$ is
\[
\frac{Q(s)}{F(s)} = \frac{1}{m s^2 + b s + k}.
\]

At this point, recall the three model ``languages'' and how they connect:
\[
\text{ODE (time domain)} \;\longleftrightarrow\; \text{TF (frequency domain)},
\qquad
\text{ODE (time domain)} \;\longleftrightarrow\; \text{SS (time domain)}.
\]
Our goal now is to rewrite the ODE as a linear state-space model that we have just developed above.:
\[
\;\dot{x} = Ax + Bu,\qquad y = Cx + Du.\;
\]
For this example, we have only one input --- the force $F(t)$ --- and one output --- the position $q(t)$. The output variable is something that, as an engineer, you care about measuring or controlling. So we have
\[
u = F, \qquad y = q.
\]

How to get a state-space model from the ODE? The key idea is to define states that describe the system. Since this is a second-order ODE, we need two states. A natural physical choice is:
\[
x_1 = q \quad \text{(position)},\qquad x_2 = \dot{q} \quad \text{(velocity)}.
\]
Collect into the state vector:
\[
x =
\begin{bmatrix}
x_1\\
x_2
\end{bmatrix}.
\]
For now we also choose the output to be position:
\[
y = q = x_1.
\]

Now, note that the state-space model requires us to write $\dot{x}$ in terms of $x$ and $u$. That is, we need 
\[  
\dot{x} = \begin{bmatrix} \dot{x}_1 \\ \dot{x}_2 \end{bmatrix} = A \begin{bmatrix} x_1 \\ x_2 \end{bmatrix} + B u.
\]
So we need to find expressions for $\dot{x}_1$ and $\dot{x}_2$ first. 
\begin{popquiz}
From the state definitions, write $\dot{x}_1$ and $\dot{x}_2$ in terms of $x_1$, $x_2$, and $u=F$.
Use the ODE
\[
m\ddot{q} + b\dot{q} + kq = F(t).
\]
\popqsplit
By definition,
\[
\dot{x}_1 = \dot{q} = x_2.
\]
Also $\dot{x}_2 = \ddot{q}$. Solve the ODE for $\ddot{q}$:
\[
m\ddot{q} = F - b\dot{q} - kq
\quad\Rightarrow\quad
\ddot{q} = -\frac{k}{m}q - \frac{b}{m}\dot{q} + \frac{1}{m}F.
\]
Substitute $q=x_1$, $\dot{q}=x_2$, and $u=F$:
\[
\dot{x}_2 = -\frac{k}{m}x_1 - \frac{b}{m}x_2 + \frac{1}{m}u.
\]
\end{popquiz}

\subsection{State equation matrices: $A$ and $B$}
We want to match
\[
\dot{x} =
\begin{bmatrix}
\dot{x}_1\\
\dot{x}_2
\end{bmatrix}
=
A
\begin{bmatrix}
x_1\\
x_2
\end{bmatrix}
+
B u.
\]

Start with the ``blank template'', these are our holes to fill in (as you are starting to work with state-space models, I highly recommend that you write these blank templates down to help you organize your thoughts because matrix products can get confusing):
\[
\begin{bmatrix}
\dot{x}_1\\
\dot{x}_2
\end{bmatrix}
=
\begin{bmatrix}
\Box & \Box\\
\Box & \Box
\end{bmatrix}
\begin{bmatrix}
x_1\\
x_2
\end{bmatrix}
+
\begin{bmatrix}
\Box\\
\Box
\end{bmatrix}u.
\]

Now use the two scalar equations:
\[
\dot{x}_1 = x_2
\qquad\Rightarrow\qquad
\dot{x}_1 = 0\cdot x_1 + 1\cdot x_2 + 0\cdot u,
\]
\[
\dot{x}_2 = -\frac{k}{m}x_1 - \frac{b}{m}x_2 + \frac{1}{m}u.
\]

So the filled matrices are:
\[
A =
\begin{bmatrix}
0 & 1\\[4pt]
-\dfrac{k}{m} & -\dfrac{b}{m}
\end{bmatrix},
\qquad
B =
\begin{bmatrix}
0\\[4pt]
\dfrac{1}{m}
\end{bmatrix}.
\]

\subsection{Output equation: $C$ and $D$ matrices}
We chose $y=q=x_1$. Match
\[
y = Cx + Du.
\]
Start with the ``blank template'':
\[
y =
\begin{bmatrix}
\Box & \Box
\end{bmatrix}
\begin{bmatrix}
x_1\\
x_2
\end{bmatrix}
+
\begin{bmatrix}
\Box
\end{bmatrix}u.
\]
Since $y=x_1$, we have
\[
y = 1\cdot x_1 + 0\cdot x_2 + 0\cdot u,
\]
so
\[
C = \begin{bmatrix} 1 & 0 \end{bmatrix},
\qquad
D = \begin{bmatrix} 0 \end{bmatrix}.
\]

\subsection{Unique or not?}
State-space is a representation, so $A,B,C,D$ are \emph{not unique}. Let us try a different state definition and see if it leads to different $A,B$ matrices.
For example, one straightforward way to change the definitions is to scale or mix the states. Let us try:
\[
\tilde{x}_1 = 2q,\qquad \tilde{x}_2 = \dot{q}.
\]
This will change the numerical entries in the matrices because the coordinates changed, even though the underlying physics did not. More formally, for an invertible change of coordinates $\tilde{x}=Tx$ (with $T$ invertible), the same dynamics can be written with a different pair $(\tilde{A},\tilde{B})`$.

Also, if instead of position you care about velocity, you could define
\[
y = \dot{q} = x_2
\quad\Rightarrow\quad
C = \begin{bmatrix}0 & 1\end{bmatrix},\;\; D=0.
\]
Or you might care about a combination like $y = q + \alpha \dot{q}$, giving
\[
C = \begin{bmatrix}1 & \alpha\end{bmatrix},\;\; D=0.
\]

Finally, sometimes the output can include the input directly:
\[
y = Cx + Du,\qquad D\neq 0.
\]
This happens when the measured quantity depends instantaneously on the input. For example, if an actuator force is measured through a sensor whose reading includes a direct feedthrough term, or if the output is defined as a combination of state and applied force, then you can have the case above.

What we have just built (not the variants, the fully built matrices $A$, $B$, $C$, $D$ above) is the \textbf{chain of integrators form} of state-space representation.
The choice
\[
x_1=q,\qquad x_2=\dot{q}
\]
produces a structure where the first state is the integral of the second:
\[
\dot{x}_1 = x_2.
\]
This is why this canonical construction is often called the \emph{chain of integrators} form: higher derivatives become downstream states, and the top states are obtained by integrating.

\subsection{Connections between TF and SS}
To connect the state space model back to transfer functions, let us look at the numerical values. Use the following values for the parameters:
\[
m=\frac{1}{6},\qquad b=\frac{5}{6},\qquad k=1.
\]
Then
\[
\frac{Q(s)}{F(s)}
=
\frac{1}{ms^2+bs+k}
=
\frac{1}{\frac{1}{6}s^2+\frac{5}{6}s+1}
=
\frac{6}{s^2+5s+6}.
\]
So the denominator polynomial is
\[
s^2+5s+6.
\]

With these same parameters, the state matrix becomes
\[
A=
\begin{bmatrix}
0 & 1\\[2pt]
-\dfrac{k}{m} & -\dfrac{b}{m}
\end{bmatrix}
=
\begin{bmatrix}
0 & 1\\
-6 & -5
\end{bmatrix}.
\]

The eigenvalues of $A$ are the roots of its characteristic equation:
\[
\det(\lambda I - A)=0.
\]
Compute:
\[
\lambda I - A
=
\begin{bmatrix}
\lambda & -1\\
6 & \lambda+5
\end{bmatrix}
\quad\Rightarrow\quad
\det(\lambda I-A)=\lambda(\lambda+5)+6.
\]
So
\[
\lambda^2+5\lambda+6=0.
\]

Key observation: this is the same polynomial as the TF denominator (up to swapping $s$ with $\lambda$).
This is not a coincidence: for a realization of an LTI system, the dynamics of the homogeneous system $\dot{x}=Ax$
govern the natural response, and those modes are captured by the eigenvalues of $A$. The transfer function poles
describe the same natural modes.

So, we have the important equivalence:
\[
\text{eigenvalues of }A \;=\; \text{roots of characteristic equation} \;=\; \text{poles of the TF}
\]
and these correspond to the modes of the system and its natural response.

\section{In-class activity: sketching step and impulse responses}
Next, we will do a short in-class activity where you sketch a nominal step response and impulse response for three real-life second-order systems (see last page). The goal is to connect the second order dynamics to what you expect to see in reality (overshoot, oscillations, settling, etc.). We will follow-up on this activity next lecture when we formally define step and impulse responses.
 
\section{Next lecture preview}
Next time:
\begin{itemize}
  \item step response and impulse response,
  \item how to compute these using transfer functions,
  \item performance metrics from responses: overshoot, settling time, rise time, steady-state value.
\end{itemize}

\newpage
\scriptsize

\noindent
\textbf{EE 122: Feb 2 In-class activity on step and impulse response}
\hfill

\textbf{Names:} \rule{2.9in}{0.4pt}

\vspace{0.4em}
\noindent
\vspace{0.6em}

% --- Three panels ------------------------------------------------------------
\noindent
\begin{minipage}[t]{0.32\textwidth}
\textbf{Example 1: Car suspension system}

\vspace{0.3em}
\textbf{Step Input:} The car drives onto a raised curb and then stays on that new road height. The suspension compresses to a new equilibrium height and settles there.

\vspace{0.4em}
\textbf{Impulse Input:} The car hits a pothole or bump very quickly and then the road returns to normal. The suspension receives a sudden jolt and then the car body oscillates and settles back to its original height.

\vspace{0.7em}
\textbf{Sketch nominal step response:}
\vspace{0.2em}

\stepaxes

\vspace{0.8em}
\textbf{Sketch nominal impulse response:}
\vspace{0.2em}

\impaxes
\end{minipage}
\hfill
\vrule width 0.8pt
\hfill
\begin{minipage}[t]{0.32\textwidth}
\textbf{Example 2: Hydraulic door closer on a heavy door}

\vspace{0.3em}
\textbf{Step Input:} You open the door to a fixed angle and hold it there. When you release it, the closer drives the door toward the closed position and it settles shut.

\vspace{0.4em}
\textbf{Impulse Input:} Someone gives the door a quick shove and immediately lets go. The door briefly moves faster than usual, then the closer resists motion and the door returns toward closed with decreasing oscillations.

\vspace{0.7em}
\textbf{Sketch nominal step response:}
\vspace{0.2em}

\stepaxes

\vspace{0.8em}
\textbf{Sketch nominal impulse response:}
\vspace{0.2em}

\impaxes
\end{minipage}
\hfill
\vrule width 0.8pt
\hfill
\begin{minipage}[t]{0.32\textwidth}
\textbf{Example 3: Mass hanging from a spring (with damping)}

\vspace{0.3em}
\textbf{Step Input:} You suddenly add a weight to the spring and keep it attached. The mass moves to a new equilibrium extension and settles there after transient motion.

\vspace{0.4em}
\textbf{Impulse Input:} You tap the mass downward quickly and let go. The mass gets a sudden kick and then oscillates about equilibrium with decreasing amplitude until it comes to rest.

\vspace{0.7em}
\textbf{Sketch nominal step response:}
\vspace{0.2em}

\stepaxes

\vspace{0.8em}
\textbf{Sketch nominal impulse response:}
\vspace{0.2em}

\impaxes
\end{minipage}

\vspace{0.6em}
\noindent

\end{document}
