\makeatletter
\def\input@path{{../styles/}{../../styles/}{../../../styles/}{../}{../../}{../../../}}
\makeatother
\documentclass{ee122_notes}
% macros.tex - Course meta information
\renewcommand{\course}{EE 122} % with a space
\renewcommand{\coursetitle}{Introduction to Control Systems}
\renewcommand{\instructor}{Ayush Pandey}
\renewcommand{\student}{Name: }

\renewcommand{\semester}{Spring 2026}
\date{\semester} % this sets the LaTeX date field safely

% Problem set number
\renewcommand{\psetnum}{1}

% Release / due — use renewcommand because package provides empties
\renewcommand{\releasedate}{January 20, 2026}
\renewcommand{\duedate}{May 14, 2026}

% The following packages can be found on http:\\www.ctan.org
% \usepackage{graphics} % for pdf, bitmapped graphics files
%\usepackage{epsfig} % for postscript graphics files
%\usepackage{mathptmx} % assumes new font selection scheme installed
%\usepackage{times} % assumes new font selection scheme installed
\usepackage{amsmath} % assumes amsmath package installed
\usepackage{amssymb,mathtools}  % assumes amsmath package installed
\usepackage{xcolor}
\usepackage{pgfplots,subcaption}
\usepackage[hidelinks]{hyperref}
\usepackage{verbatim}
\usepackage{graphicx}
\usepackage{listings}
\usepackage{fancyhdr}
% \usepackage{geometry}
\usepackage{siunitx}
\usepackage[most]{tcolorbox}
\usepackage{enumitem}
\usepackage{environ}
\usepackage{pifont}
% -------- listings (Python) ----------
\lstdefinestyle{py}{
  language=Python,
  basicstyle=\ttfamily\small,
  keywordstyle=\color{blue!60!black}\bfseries,
  commentstyle=\color{green!40!black},
  stringstyle=\color{orange!60!black},
  showstringspaces=false,
  columns=fullflexible,
  frame=single,
  framerule=0.3pt,
  numbers=left,
  numberstyle=\tiny,
  xleftmargin=1em,
  tabsize=2,
  breaklines=true,
}

\usepackage[american]{circuitikz}
\usepackage{tikz}
\usetikzlibrary{arrows.meta,positioning,calc,angles,quotes}
\tikzset{
  >={Latex[length=2.2mm]},
  block/.style={draw, thick, rectangle, minimum height=10mm, minimum width=24mm, align=center},
  gain/.style={block, minimum width=14mm},
  sum/.style={draw, thick, circle, inner sep=0pt, minimum size=6mm},
  conn/.style={-Latex, thick},
}
\usepackage{caption}    
\usepackage{lscape}
\usepackage{soul}
\usepackage{physics}
\usepackage{hyperref}
\hypersetup{
    colorlinks=true,
    linkcolor=blue,
    filecolor=magenta,      
    urlcolor=blue,
    pdftitle={week1_notes},
    pdfpagemode=FullScreen,
}
%\usepackage{float} 

%\usepackage[demo]{graphicx}
\pgfplotsset{compat=1.18}
% \usepgfplotslibrary{fillbetween}

\newsavebox{\measurebox}

\let\proof\relax\let\endproof\relax


\def\abs#1{\left\lvert#1\right\rvert}
\let\proof\relax
\let\endproof\relax
\usepackage{amsthm}
\usepackage{accents}
\usepackage{relsize}
\newcommand{\ubar}[1]{\underaccent{\bar}{#1}}
\newtheorem{theorem}{Theorem}
\newtheorem{corollary}{Corollary}[theorem]
\newtheorem{lemma}{Lemma}
\newtheorem{proposition}{Proposition}
\newtheorem{statement}{Statement}

\theoremstyle{definition}
\newtheorem{definition}{Definition}
 
\theoremstyle{remark}
\newtheorem*{remark}{Remark}
\theoremstyle{remark}
\newtheorem*{claim}{Claim}
\setlength{\parindent}{0cm}
\newenvironment{nalign}{
    \begin{equation}
    \begin{aligned}
}{
    \end{aligned}
    \end{equation}
    \ignorespacesafterend
} 

\renewcommand{\releasedate}{January 26, 2026}
\begin{document}

\section*{EE 122/ME 141 Week 2, Lecture 1 (Spring 2026)}
\subsection*{Instructor: \instructor}
\subsection*{Date: \releasedate}
\section{Announcements and Recap}
\begin{enumerate} 
    \item Problem set 1 is due on Wed Jan 28 at 11:59pm. Please submit one complete PDF (with code and plots) on Gradescope.
    \item Lab 1 is due during the Wed/Thu lab sessions. You must finish the pre-lab work before coming to the lab.
\end{enumerate}

\subsection{A short introduction to control systems glossary}
\begin{figure}[h]
  \centering
  \begin{blockdiagram}
    \reference{$r(t)$}
    \controller[Controller]{$e(t)$}{$u(t)$}
    \system[Plant]{}{$y(t)$}
    \feedback[Sensor]{}{}
  \end{blockdiagram}
  \caption{A typical feedback control system architecture. The reference $r(t)$ describes the desired output behavior, the controller computes control inputs (or control actions) $u(t)$, and the system (or plant) produces outputs $y(t)$ that are measured by sensors to produce measurements $y_m(t)$. The error signal $e(t)$ is the difference between the reference and the measured output. Noise can affect the sensor measurements. Disturbances can affect the system dynamics.}
  \label{fig:control_block_diagram}
\end{figure}

To recap, we discussed a typical control system architecture that typically consists of a dynamical system that we intend to control, a controller system that generates the control actions, and sensors that measure the system outputs. The controller uses the sensor measurements to compute control actions that are applied to the system to achieve desired behavior. This is visualized as a block diagram in Figure~\ref{fig:control_block_diagram}.

We use many terms when studying control systems so it's worth creating a list of many of the key terms:
\begin{enumerate}
    \item A control system: Refers to the overall system --- one that uses control actions to influence the behavior of a physical system. This includes the physical system, the controller, and the sensors (and possibly, other systems).
    \item A dynamical system / A physical system / System / A plant: We will use these terms interchangeably to refer to the physical system that we want to control. Note that ``dynamical system'' is a broader term that can also be used to refer to any system that evolves over time, so a controller is also a dynamical system. So, the meaning of ``dynamical system'' depends on the context and therefore, we will often call it the ``physical system'', or sometimes for brevity, just ``system'', where it will be clear from context what we are referring to.
    \item Controller: A system that computes control actions (inputs) to the physical system based on sensor measurements and possibly other information (like a reference command and other information). This is where the logic / algorithm for how to compute control inputs $u$ lives. Usually, controllers are implemented using computers but circuits and mechanical controllers can also be designed and in fact are commonly used. Another commonly used term for the controller is a ``compensator'' because this is a system that compensates for the undesirable aspects in the physical system.
    \item Control input / control law: This is the signal, usually denoted by $u(t)$, that is generated by the controller, that is, the output of the controller, and applied to the physical system to influence its behavior. Control law is an equation that describes $u$ such as $u = k(x)$ is a control law where $k$ is some function of the system state $x$. Designing ``good'' control laws (and understanding what ``good'' means) is the main goal of this course.
    \item Reference command / reference input: This is the desired behavior for the system output, usually denoted by $r(t)$. For example, in a cruise control system, the reference command is the desired speed of the vehicle. The controller's job is to ensure that the system output $y(t)$ tracks (follows) the reference command $r(t)$ as closely as possible.
    \item Sensor / measurement: A sensor is a device that measures some aspect of the output(s) of the physical system. We may not be able to measure all of the outputs directly, and the measurements may be noisy or imperfect. The sensor output is usually denoted by $y_m(t)$.
    \item Disturbance: A disturbance is an external input to the physical system that affects its behavior but is not controlled by the controller. Usually denoted by $d(t)$.
\end{enumerate}

To go any further, we must be able to mathematically describe each of the blocks in the block diagram in Figure~\ref{fig:control_block_diagram}. Therefore, we start by discussing the theory of modeling dynamical systems and analyzing them using transfer functions --- a frequency domain representation of a dynamical system.
\section{Dynamical System Modeling}
A control system uses a model to predict how a physical system will evolve. A model is a mathematical description of the system dynamics (often an ODE or state-space model). The model predicts how inputs affect outputs. For example, when we use Kirchoff's laws to model an electrical circuit, we write a differential equation that relates the output voltage to the input source voltage and other inputs.

\section{Time domain and frequency domain}
The most direct and intuitive way to describe a dynamical system is in the time domain, where we write equations that relate inputs and outputs as functions of time. However, signals (like inputs and outputs of a system) can get quite complicated in the time domain. We aim for general set of methods that can be applied across a wide variety of systems and signals. Therefore, we often work in the frequency domain, where what we find is that the system behavior can be described in a way that allows us to apply a set of mathematical tools reliably. Having said that, we will always strive to keep the time domain picture alive to not forego the intuition that comes with it.

Another way to look at frequency domain is the following hypothesis: every signal can be represented as a combination of exponentials and sinusoids. Therefore, if we define the system that only describes the constants associated with these exponentials (the decay / growth rates: will be called poles) and the frequencies of the sinusoids, then we end up with a much simpler analysis of the system. If all of this text reads confusing, do not worry! As we set the foundations up, you will develop your frequency domain intuition as well. 

The most basic tool that lets us compute a frequency domain representation of a signal is the Fourier transform. 
\subsection{Fourier transform recap}
For a signal $x(t)$, the (continuous-time) Fourier transform is
\[
X(\omega) = \int_{-\infty}^{\infty} x(t)e^{-j\omega t}\,dt,
\qquad
x(t)=\frac{1}{2\pi}\int_{-\infty}^{\infty}X(j\omega)e^{j\omega t}\,d\omega,
\]
where the second equation is the inverse Fourier transform that can be used to recover $x(t)$ from $X(\omega)$. Let us try to apply the Fourier transform to a problem to see if it tells us anything meaningful. 

\subsection{Example: decaying exponential}
Consider the causal decaying exponential
\[
y(t)=e^{-at}u(t), \qquad a>0.
\]
Here, $u(t)$ is the unit step function that is zero for $t<0$ and one for $t\geq 0$ shown in Figure~\ref{fig:unit_step}.
\begin{figure}[h]
    \centering
    \begin{tikzpicture}
        \begin{axis}[
            axis lines=middle,
            xlabel={$t$},
            ylabel={$u(t)$},
            xtick={0},
            ytick={0,1},
            ymin=-0.5, ymax=1.5,
            xmin=-2, xmax=2,
            width=8cm,
            height=5cm,
            domain=-2:2,
            samples=100,
        ]
        \addplot[thick,blue] {x >= 0 ? 1 : 0};
        \end{axis}
    \end{tikzpicture}
    \caption{Unit step function $u(t)$.}
    \label{fig:unit_step}
\end{figure}
It is clear from the figure that multiplying any signal with $u(t)$ will simply ``activate'' the signal at $t=0$ and make it zero for $t<0$. When we have the step function inside an integration, we can simply change the limits of the integral so that we do not integrate for any time before $t=0$ and for any time after $t=0$, we replace $u(t)$ with $1$. Apply this to solve the pop-quiz below.

\begin{popquiz}
Compute the Fourier transform of $y(t)=e^{-2t}u(t)$. Then compute $|Y(\omega)|$, sketch it, and describe what happens to the magnitude as $|\omega|\to\infty$.
\popqsplit
We can evaluate the integral in the Fourier transform equation as 
\[
Y(\omega) = \int_{0}^{\infty} e^{-2t}e^{-j\omega t}\,dt
=\int_{0}^{\infty} e^{-(2+j\omega)t}\,dt.
= \left[ \frac{e^{-(2+j\omega)t}}{-(2+j\omega)} \right]_{0}^{\infty}
= \frac{1}{2+j\omega}.
\]
\[
Y(j\omega)=\frac{1}{2+j\omega}.
\]
The magnitude is (recall that, to compute the absolute value of a complex number $a+jb$, we use $\sqrt{a^2+b^2}$)
\[
|Y(j\omega)|=\frac{1}{\sqrt{4+\omega^2}},
\]
so $|Y(j\omega)|\to 0$ as $|\omega|\to\infty$ and decays like $1/|\omega|$.
\end{popquiz}

From the solution to the pop-quiz above, we can see that for the decaying exponential, transforming the signal into the frequency domain gave us some useful information that was not apparent from the time-domain representation. In particular, we can see that the signal contains most prominently low-frequency components (because the magnitude is large for small $\omega$) and the high-frequency components are attenuated (because the magnitude decays to zero as $|\omega|$ increases). A real-world significance of this fact is that this signal is ``smooth'' and does not have ``ringy/oscillatory'' behavior that is typical of a high frequency signal (think of a sine wave with a very high frequency --- those types of behavior are not present in the signal whereas, sine waves with lower frequencies do add up to form the original signal).

Let us continue to apply the Fourier transform. This time a different signal --- an exponentially growing one.
\begin{popquiz}
Compute the Fourier transform of $y(t)=e^{2t}u(t)$ by evaluating the integral
\[
Y(\omega)=\int_{0}^{\infty} e^{2t}e^{-j\omega t}\,dt.
\]
Can you compute this integral? Why or why not? What is the Fourier transform? How does it behave? 
\popqsplit
We can rewrite the integrand as $e^{(2-j\omega)t}$. Its magnitude is $|e^{(2-j\omega)t}|=e^{2t}$, which grows without bound as $t\to\infty$. Therefore the integral
\[
\int_{0}^{\infty} e^{(2-j\omega)t}\,dt
\]
diverges for every real $\omega$. The Fourier transform does not exist for $e^{2t} u(t)$! 
\end{popquiz}

So, what is going on here!? Why did the Fourier transform work for the decaying exponential but not for the growing exponential? The answer lies in convergence of the integral --- so, do not apply Fourier transform identities blindly. An important condition for application of Fourier transform is that the integral must converge. If the signal grows too quickly, the integral may diverge for all frequencies, and the Fourier transform does not exist. What do we do then? 

The \textbf{key idea} here is that we do want some kind of frequency domain transformation for a much wider class of signals. If we already \textit{give up} analyzing signals that grow without bound then we are giving up on many real-world systems, that need to be stabilized and controlled! So, the whole purpose of this class is defeated if we cannot analyze / study signals that grow without bound. So, we present an \textit{artificial} modification to the original signal that allows us to analyze it in a new kind of frequency domain. 

The main problem with the signal is that it is exponentially growing. So, we can make it \textit{not} do that by multiplying a sufficiently fast decaying exponential to it so that overall the signal decays. Let us try this next. 
\subsection{Fixing convergence by exponential weighting}
Define the artificially modified new signal as $\bar{y}(t) = y(t)e^{-\sigma t}$. Now, consider the Fourier transform of the weighted signal, $\bar{y}(t)$:
\[
\int_{-\infty}^{\infty} \big(y(t)e^{-\sigma t}\big)e^{-j\omega t}\,dt
=
\int_{-\infty}^{\infty} y(t)e^{-(\sigma + j\omega)t}\,dt.
\]
If $\sigma$ is chosen large enough (in the appropriate direction), the exponential factor can overcome growth in $y(t)$ and make the integral converge. We define a new complex frequency variable $s := \sigma + j\omega$, and rewrite the integral as
\[
X(s)=\int_{-\infty}^{\infty} y(t)e^{-st}\,dt.
\]
This is the bilateral Laplace transform! The Fourier transform is recovered as the special case $\sigma=0$ (so $s=j\omega$), when that choice lies in the set of values for which the integral converges.

So, effectively, what we have done is --- Fourier transform, and the frequency domain (the usual one with real frequencies, $\omega$) did not exist for unstable signals. Therefore, we have defined a new frequency domain (with complex frequencies, $s$) where we can analyze a much wider class of signals. The real frequencies are not lost, they are still there as a special case (when $\sigma=0$).
\subsection{Unilateral Laplace transform}
In control, many signals are causal: they are zero before a chosen time origin, often taken as $t=0$. That is, signals do not exist for negative time. For example, when we conduct an experiment, we denote the start of the experiment as $t=0$ and the system is at rest before that time. In such cases, it is natural to define the unilateral (or one-sided) Laplace transform: 
\[
X^+(s)=\int_{0^-}^{\infty} x(t)e^{-st}\,dt.
\]

We need to be careful about the convergence of the Laplace transform. As we saw above, we were able to define the Laplace transform for the signal $y(t) = e^{2t}u(t)$ only for $\Re{s} = \sigma > 2$. This is the region of convergence (ROC) for that signal. In general, for any signal, when writing the Laplace transform, one must be careful about its region of convergence: that is, the set of $s$ values for which the Laplace transform integral converges. The ROC matters because it distinguishes different time-domain signals that can have the same algebraic form in $s$.

\begin{popquiz}
For $x(t)=e^{at}u(t)$ with $a>0$, compute the bilateral Laplace transform and determine the condition on $s$ for which it converges. Then state whether the Fourier transform exists.
\popqsplit
For $x(t)=e^{at}u(t)$,
\[
X(s)=\int_{0}^{\infty} e^{at}e^{-st}\,dt=\int_{0}^{\infty} e^{-(s-a)t}\,dt=\frac{1}{s-a}.
\]
The integral converges when $\mathrm{Re}(s-a)>0$, so the ROC is $\mathrm{Re}(s)>a$.
The Fourier transform would correspond to $s=j\omega$ (meaning $\mathrm{Re}(s)=0$), but $0>a$ is false since $a>0$. Therefore the Fourier transform does not exist.
\end{popquiz}

\subsection{A concluding note}
We observe that many system responses are sums of exponentials and sinusoids. Therefore, the analysis above turns out to be very useful. Laplace transforms handle these naturally and make differentiation and integration algebraic, which is why Laplace is the main tool in defining transfer functions. Let's end this section with another practice problem.

\begin{popquiz}
Compute the unilateral Laplace transform of $x(t)=e^{3t}u(t)$ and state the condition on $s$ for which it converges.
\popqsplit
\[
X^+(s)=\int_0^\infty e^{3t}e^{-st}\,dt=\int_0^\infty e^{-(s-3)t}\,dt=\frac{1}{s-3},
\]
which converges when $\mathrm{Re}(s-3)>0$, i.e. $\mathrm{Re}(s)>3$.
\end{popquiz}

\section{From ODEs to transfer functions}
Transfer functions arise from linear differential equations by taking a Laplace transform (typically under zero initial conditions). This gives an input--output relationship in the frequency-like variable $s$.

\subsection{One-dimensional ODE example}
Consider a first-order system
\[
\dot{y}(t) + a\,y(t) = b\,u(t).
\]
Taking the Laplace transform with zero initial condition (refer to Laplace transform tables) gives
\[
sY(s) + aY(s) = bU(s),
\qquad
\frac{Y(s)}{U(s)}=\frac{b}{s+a}.
\]
This is a first-order transfer function. The pole is at $s=-a$.

\begin{popquiz}
A system is modeled by $\dot{y}(t)+5y(t)=2u(t)$ with $y(0)=0$. Compute the transfer function $G(s)=Y(s)/U(s)$. Then compute $Y(s)$ for a step input $u(t)=u(t)$ (unit step).
\popqsplit
With $y(0)=0$,
\[
(s+5)Y(s)=2U(s)\quad\Rightarrow\quad G(s)=\frac{2}{s+5}.
\]
For a unit step, $U(s)=1/s$, so
\[
Y(s)=G(s)U(s)=\frac{2}{s(s+5)}.
\]
\end{popquiz}

\subsection{Two-dimensional ODE example}
A standard second-order model is
\[
\ddot{y}(t) + 2\zeta\omega_n \dot{y}(t) + \omega_n^2 y(t) = k\,u(t).
\]
With zero initial conditions, the Laplace transform gives
\[
\left(s^2 + 2\zeta\omega_n s + \omega_n^2\right)Y(s)=k\,U(s),
\qquad
G(s)=\frac{Y(s)}{U(s)}=\frac{k}{s^2 + 2\zeta\omega_n s + \omega_n^2}.
\]
This denominator polynomial is the characteristic polynomial. Its roots are the poles.
\section{Next steps}
In the next lecture, we will discuss how to analyze transfer functions to understand system behavior in both time and frequency domains. Particularly, we will discuss common metrics associated with second-order transfer functions.
\end{document}
