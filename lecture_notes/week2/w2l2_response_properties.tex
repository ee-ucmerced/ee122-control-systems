\makeatletter
\def\input@path{{../styles/}{../../styles/}{../../../styles/}{../}{../../}{../../../}}
\makeatother
\documentclass{ee122_notes}
% macros.tex - Course meta information
\renewcommand{\course}{EE 122} % with a space
\renewcommand{\coursetitle}{Introduction to Control Systems}
\renewcommand{\instructor}{Ayush Pandey}
\renewcommand{\student}{Name: }

\renewcommand{\semester}{Spring 2026}
\date{\semester} % this sets the LaTeX date field safely

% Problem set number
\renewcommand{\psetnum}{1}

% Release / due — use renewcommand because package provides empties
\renewcommand{\releasedate}{January 20, 2026}
\renewcommand{\duedate}{May 14, 2026}

% The following packages can be found on http:\\www.ctan.org
% \usepackage{graphics} % for pdf, bitmapped graphics files
%\usepackage{epsfig} % for postscript graphics files
%\usepackage{mathptmx} % assumes new font selection scheme installed
%\usepackage{times} % assumes new font selection scheme installed
\usepackage{amsmath} % assumes amsmath package installed
\usepackage{amssymb,mathtools}  % assumes amsmath package installed
\usepackage{xcolor}
\usepackage{pgfplots,subcaption}
\usepackage[hidelinks]{hyperref}
\usepackage{verbatim}
\usepackage{graphicx}
\usepackage{listings}
\usepackage{fancyhdr}
% \usepackage{geometry}
\usepackage{siunitx}
\usepackage[most]{tcolorbox}
\usepackage{enumitem}
\usepackage{environ}
\usepackage{pifont}
% -------- listings (Python) ----------
\lstdefinestyle{py}{
  language=Python,
  basicstyle=\ttfamily\small,
  keywordstyle=\color{blue!60!black}\bfseries,
  commentstyle=\color{green!40!black},
  stringstyle=\color{orange!60!black},
  showstringspaces=false,
  columns=fullflexible,
  frame=single,
  framerule=0.3pt,
  numbers=left,
  numberstyle=\tiny,
  xleftmargin=1em,
  tabsize=2,
  breaklines=true,
}

\usepackage[american]{circuitikz}
\usepackage{tikz}
\usetikzlibrary{arrows.meta,positioning,calc,angles,quotes}
\tikzset{
  >={Latex[length=2.2mm]},
  block/.style={draw, thick, rectangle, minimum height=10mm, minimum width=24mm, align=center},
  gain/.style={block, minimum width=14mm},
  sum/.style={draw, thick, circle, inner sep=0pt, minimum size=6mm},
  conn/.style={-Latex, thick},
}
\usepackage{caption}    
\usepackage{lscape}
\usepackage{soul}
\usepackage{physics}
\usepackage{hyperref}
\hypersetup{
    colorlinks=true,
    linkcolor=blue,
    filecolor=magenta,      
    urlcolor=blue,
    pdftitle={week1_notes},
    pdfpagemode=FullScreen,
}
%\usepackage{float} 

%\usepackage[demo]{graphicx}
\pgfplotsset{compat=1.18}
% \usepgfplotslibrary{fillbetween}

\newsavebox{\measurebox}

\let\proof\relax\let\endproof\relax


\def\abs#1{\left\lvert#1\right\rvert}
\let\proof\relax
\let\endproof\relax
\usepackage{amsthm}
\usepackage{accents}
\usepackage{relsize}
\newcommand{\ubar}[1]{\underaccent{\bar}{#1}}
\newtheorem{theorem}{Theorem}
\newtheorem{corollary}{Corollary}[theorem]
\newtheorem{lemma}{Lemma}
\newtheorem{proposition}{Proposition}
\newtheorem{statement}{Statement}

\theoremstyle{definition}
\newtheorem{definition}{Definition}
 
\theoremstyle{remark}
\newtheorem*{remark}{Remark}
\theoremstyle{remark}
\newtheorem*{claim}{Claim}
\setlength{\parindent}{0cm}
\newenvironment{nalign}{
    \begin{equation}
    \begin{aligned}
}{
    \end{aligned}
    \end{equation}
    \ignorespacesafterend
} 

\renewcommand{\releasedate}{January 28, 2026}
\begin{document}

\section*{EE 122/ME 141 Week 2, Lecture 2 (Spring 2026)}
\subsection*{Instructor: \instructor}
\subsection*{Date: \releasedate}
\section{Goals}

\subsection{Homogeneous solution}
The homogeneous solution is the response when the input is zero. For a second-order system,
\[
\ddot{y} + 2\zeta\omega_n \dot{y} + \omega_n^2 y = 0,
\]
the homogeneous solution is determined by the characteristic equation
\[
s^2 + 2\zeta\omega_n s + \omega_n^2 = 0.
\]

\subsection{Particular solution}
The particular solution is any one solution that matches the forcing input. For example, a constant (step) input often yields a constant steady-state output, while a sinusoidal input yields a sinusoidal steady-state output at the same frequency, with an amplitude and phase determined by the transfer function.

\subsection{Impulse and step as standard test inputs}
Two canonical inputs are:
\[
u(t)=\delta(t) \quad \text{(impulse)}, \qquad u(t)=u(t) \quad \text{(unit step)}.
\]
The impulse response is the output to $\delta(t)$ with zero initial conditions. The step response is the output to a unit step with zero initial conditions. These are forced responses. In contrast, a free-decay pendulum release is a zero-input response driven by nonzero initial conditions.

\begin{popquiz}
A second-order system has transfer function $G(s)=\frac{k}{s^2+2\zeta\omega_n s+\omega_n^2}$. Write an expression for $Y(s)$ for (i) an impulse input and (ii) a unit-step input. Do not invert the transforms.
\popqsplit
For an impulse, $U(s)=1$, so
\[
Y(s)=G(s)=\frac{k}{s^2+2\zeta\omega_n s+\omega_n^2}.
\]
For a unit step, $U(s)=1/s$, so
\[
Y(s)=\frac{k}{s\left(s^2+2\zeta\omega_n s+\omega_n^2\right)}.
\]
\end{popquiz}

\section{Natural frequency, damping, and damped frequency}
Second-order dynamics are everywhere: suspensions, doors, elevators, robots, and even sensor filtering. The parameters $\omega_n$ and $\zeta$ are a compact way to describe what the system does.

\subsection{Natural frequency $\omega_n$}
\textbf{Interpretation.} Roughly, $\omega_n$ sets the time scale of oscillation. For lightly damped systems, the oscillations are close to frequency $\omega_n$.

\textbf{Example interpretation.} Two car suspensions can have the same damping ratio but different $\omega_n$. The one with larger $\omega_n$ responds faster and oscillates more rapidly.

\subsection{Damping and damping ratio $\zeta$}
\textbf{Interpretation.} Damping describes how quickly oscillations die out. The dimensionless damping ratio $\zeta$ classifies the response:
\[
0<\zeta<1 \ \text{underdamped (oscillatory)}, \qquad
\zeta=1 \ \text{critically damped}, \qquad
\]
\[
\zeta>1 \ \text{overdamped (non-oscillatory)}.
\]

\textbf{Example interpretation.} A door closer is designed to avoid oscillation. Its effective damping ratio is typically at or above $1$ so the door returns smoothly without bouncing.

\subsection{Damped frequency $\omega_d$ and its dependence on $\zeta$}
For the characteristic equation
\[
s^2 + 2\zeta\omega_n s + \omega_n^2 = 0,
\]
the roots are
\[
s = -\zeta\omega_n \pm \omega_n\sqrt{\zeta^2-1}.
\]
When $0<\zeta<1$, the roots are complex:
\[
s = -\zeta\omega_n \pm j\omega_n\sqrt{1-\zeta^2}.
\]
The oscillation frequency is the imaginary part:
\[
\omega_d = \omega_n\sqrt{1-\zeta^2}.
\]
As damping increases (larger $\zeta$), the oscillation frequency decreases.

\begin{popquiz}
A system has $\omega_n=10$ rad/s. Compute $\omega_d$ for (i) $\zeta=0.1$, (ii) $\zeta=0.6$, and (iii) $\zeta=1.2$. State which cases oscillate.
\popqsplit
For $0<\zeta<1$, $\omega_d=\omega_n\sqrt{1-\zeta^2}$.
(i) $\omega_d=10\sqrt{1-0.01}=10\sqrt{0.99}\approx 9.95$ rad/s, oscillatory.
(ii) $\omega_d=10\sqrt{1-0.36}=10\sqrt{0.64}=8$ rad/s, oscillatory.
(iii) $\zeta>1$ means no oscillation; the poles are real and the response is overdamped.
\end{popquiz}

\section{Poles as roots of the characteristic equation}
For a transfer function
\[
G(s)=\frac{N(s)}{D(s)},
\]
the poles are the roots of $D(s)$. Poles determine the natural response and the transient behavior. In the second-order case,
\[
D(s)=s^2+2\zeta\omega_n s+\omega_n^2,
\]
so the pole locations encode both the decay rate (real part) and oscillation frequency (imaginary part). This is why identifying $\zeta$ and $\omega_n$ from a free-decay experiment is still a meaningful modeling task: it identifies the dominant poles of the dynamics.
\end{document}