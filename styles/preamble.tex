% The following packages can be found on http:\\www.ctan.org
% \usepackage{graphics} % for pdf, bitmapped graphics files
%\usepackage{epsfig} % for postscript graphics files
%\usepackage{mathptmx} % assumes new font selection scheme installed
%\usepackage{times} % assumes new font selection scheme installed

\usepackage{amsmath} % assumes amsmath package installed
\usepackage{amssymb,mathtools}  % assumes amsmath package installed

\usepackage{xcolor}
\usepackage{pgfplots,subcaption}
\usepackage{verbatim}
\usepackage{graphicx}
\usepackage{listings}
\usepackage{geometry}  
\usepackage{siunitx}
\usepackage[most]{tcolorbox}
\usepackage{enumitem}
\usepackage{environ}
\usepackage{pifont}

\usepackage{booktabs}
\usepackage{array}
\usepackage{multirow}
\usepackage{tikz}
\usepackage{comment}
\usepackage{fancyhdr}
\newtcolorbox{answerbox}[1][]{%
  colback=white,
  colframe=black,
  boxrule=0.6pt,
  left=6pt,right=6pt,top=4pt,bottom=4pt,
  #1
}
\newcommand{\ansline}[1]{\begin{answerbox}[height=#1]\end{answerbox}}

% Helpful notation
\newcommand{\dd}{\,\mathrm{d}}
\newcommand{\rad}{\mathrm{rad}}
\usepackage[hidelinks]{hyperref}
% -------- listings (Python) ----------
\lstdefinestyle{py}{
  language=Python,
  basicstyle=\ttfamily\small,
  keywordstyle=\color{blue!60!black}\bfseries,
  commentstyle=\color{green!40!black},
  stringstyle=\color{orange!60!black},
  showstringspaces=false,
  columns=fullflexible,
  frame=single,
  framerule=0.3pt,
  numbers=left,
  numberstyle=\tiny,
  xleftmargin=1em,
  tabsize=2,
  breaklines=true,
}

\usepackage[american]{circuitikz}
\usetikzlibrary{arrows.meta,positioning,calc,angles,quotes}
\tikzset{
  >={Latex[length=2.2mm]},
  block/.style={draw, thick, rectangle, minimum height=10mm, minimum width=24mm, align=center},
  gain/.style={block, minimum width=14mm},
  sum/.style={draw, thick, circle, inner sep=0pt, minimum size=6mm},
  conn/.style={-Latex, thick},
}
\usepackage{caption}    
\usepackage{lscape}
\usepackage{soul}
\usepackage{physics}
\usepackage{hyperref}
\hypersetup{
    colorlinks=true,
    linkcolor=blue,
    filecolor=magenta,      
    urlcolor=blue,
}
%\usepackage{float} 

%\usepackage[demo]{graphicx}
\pgfplotsset{compat=1.18}
% \usepgfplotslibrary{fillbetween}

\newsavebox{\measurebox}

\let\proof\relax\let\endproof\relax


\def\abs#1{\left\lvert#1\right\rvert}
\let\proof\relax
\let\endproof\relax
\usepackage{amsthm}
\usepackage{accents}
\usepackage{relsize}
\newcommand{\ubar}[1]{\underaccent{\bar}{#1}}
\newtheorem{theorem}{Theorem}
\newtheorem{corollary}{Corollary}[theorem]
\newtheorem{lemma}{Lemma}
\newtheorem{proposition}{Proposition}
\newtheorem{statement}{Statement}

\theoremstyle{definition}
\newtheorem{definition}{Definition}
 
\theoremstyle{remark}
\newtheorem*{remark}{Remark}
\theoremstyle{remark}
\newtheorem*{claim}{Claim}
\setlength{\parindent}{0cm}
\newenvironment{nalign}{
    \begin{equation}
    \begin{aligned}
}{
    \end{aligned}
    \end{equation}
    \ignorespacesafterend
} 
%========================
% EE 122: Block-diagram mini-library (fixed-layout)
% Provides TWO separate environments:
%   1) blockdiagram       : WITH summing junction
%   2) blockdiagramnosum  : WITHOUT summing junction
%
% Commands (work in both envs):
%   \reference{<label>}
%   \controller[<block text>]{<in-label>}{<out-label>}
%   \system[<block text>]{<in-label>}{<out-label>}
%   \feedback{<label>}                % unity feedback
%   \feedback[<sensor text>]{<label>} % feedback with sensor block
%
% Optional spacing keys in both envs:
%   [xC=..., xP=..., xY=..., xR=..., fbY=...]
%========================

\usepackage{xparse}

% --- Namespaced styles (won't collide with your existing styles) ---
\tikzset{
  cbd/.style={thick, font=\normalsize},
  cbd/block/.style={draw, rectangle, minimum height=12mm, minimum width=18mm, align=center},
  cbd/sum/.style={draw, circle, inner sep=0pt, minimum size=7mm},
  cbd/arrow/.style={->, >=Latex, thick},
}

% --- Layout keys (spacing only) ---
\pgfkeys{
  /blockdiagram/.is family,
  /blockdiagram/.cd,
  xC/.initial=3.2,    % controller x-position
  xP/.initial=7.2,    % plant x-position
  xY/.initial=10.2,   % output x-position
  xR/.initial=-2.2,   % reference start x-position
  fbY/.initial=-2.2,  % feedback routing depth (negative is below)
}

% --- Internal state ---
\newif\ifbdHasSum         % whether the current environment has a summing junction
\bdHasSumtrue             % default; environments set this appropriately
\newcommand{\bdCtrlOut}{}  % controller output label used on controller->plant arrow
\newcommand{\bdRefLabel}{} % reference label stored until controller exists (nosum env)

% --- Helper: robust empty test ---
\newcommand{\IfEmptyTF}[3]{%
  \edef\bdTmp{\detokenize{#1}}%
  \ifx\bdTmp\empty
    #2%
  \else
    #3%
  \fi
}

%========================
% Environment 1: WITH summing junction
%========================
\NewDocumentEnvironment{blockdiagram}{ O{} }
{
  \bdHasSumtrue
  \pgfkeys{/blockdiagram/.cd, #1}

  \begin{tikzpicture}[cbd]

    % Coordinates
    \coordinate (bdR)    at (\pgfkeysvalueof{/blockdiagram/xR},0);
    \coordinate (bdCpos) at (\pgfkeysvalueof{/blockdiagram/xC},0);
    \coordinate (bdPpos) at (\pgfkeysvalueof{/blockdiagram/xP},0);
    \coordinate (bdY)    at (\pgfkeysvalueof{/blockdiagram/xY},0);

    % Summing junction and diagram input anchor
    \node[cbd/sum] (bdSum) at (0,0) {};
    % \node at ($(bdSum.west)+(0.35,0.25)$) {\scriptsize $+$};
    % \node at ($(bdSum.south)+(0.25,0.35)$) {\scriptsize $-$};
    \coordinate (bdIn) at (bdSum.east);

    % Reset state
    \renewcommand{\bdCtrlOut}{}
    \renewcommand{\bdRefLabel}{}
}
{
  \end{tikzpicture}
}

%========================
% Environment 2: NO summing junction
%========================
\NewDocumentEnvironment{blockdiagramnosum}{ O{} }
{
  \bdHasSumfalse
  \pgfkeys{/blockdiagram/.cd, #1}

  \begin{tikzpicture}[cbd]

    % Coordinates
    \coordinate (bdR)    at (\pgfkeysvalueof{/blockdiagram/xR},0);
    \coordinate (bdCpos) at (\pgfkeysvalueof{/blockdiagram/xC},0);
    \coordinate (bdPpos) at (\pgfkeysvalueof{/blockdiagram/xP},0);
    \coordinate (bdY)    at (\pgfkeysvalueof{/blockdiagram/xY},0);

    % Placeholder input anchor (becomes controller west once controller is created)
    \coordinate (bdIn) at (0,0);

    % Reset state
    \renewcommand{\bdCtrlOut}{}
    \renewcommand{\bdRefLabel}{}
}
{
  \end{tikzpicture}
}

%========================
% Commands (work in both environments)
%========================

% Reference arrow:
% - With sum: draws into bdSum.west.
% - No sum: stores label; controller will draw the arrow to bdC.west.
\NewDocumentCommand{\reference}{ m }{%
  \ifbdHasSum
    \draw[cbd/arrow] (bdR) -- node[pos=0.5,above]{#1} (bdSum.west);
  \else
    \renewcommand{\bdRefLabel}{#1}
  \fi
}

% Controller:
% - With sum: arrow from bdIn (bdSum.east) to controller.
% - No sum: first defines bdIn=bdC.west and draws ONE reference arrow bdR->bdC.west.
\NewDocumentCommand{\controller}{ O{Controller} m m }{%
  \node[cbd/block] (bdC) at (bdCpos) {#1};

  \ifbdHasSum
    % input anchor already bdSum.east
  \else
    \coordinate (bdIn) at (bdC.west);
    % draw the stored reference arrow exactly once (if provided)
    \IfEmptyTF{\bdRefLabel}{}{%
      \draw[cbd/arrow] (bdR) -- node[pos=0.5,above]{\bdRefLabel} (bdC.west);
    }%
  \fi

  % Draw the controller input arrow (from bdIn to bdC.west)
  % In nosum mode, bdIn == bdC.west, so this would be zero-length; skip it there.
  \ifbdHasSum
    \draw[cbd/arrow] (bdIn) -- node[pos=0.5,above]{#2} (bdC.west);
  \else
    % In nosum, interpret #2 as the label for the incoming signal at controller input.
    % Place it near the reference arrow instead of drawing a second arrow.
    \IfEmptyTF{\bdRefLabel}{%
      % If no reference arrow was provided, still annotate near controller input
      \node[above] at ($(bdC.west)!0.5!(bdCpos)$) {#2};
    }{%
      % Put the controller input label slightly below the reference label to avoid overlap
      \node[below] at ($(bdR)!0.6!(bdC.west)$) {#2};
    }%
  \fi

  \renewcommand{\bdCtrlOut}{#3}
}

% System/Plant:
\NewDocumentCommand{\system}{ O{System} m m }{%
  \node[cbd/block] (bdP) at (bdPpos) {#1};

  \draw[cbd/arrow] (bdC.east) --
    node[pos=0.5,above]{\bdCtrlOut}
    node[pos=0.5,below]{#2}
    (bdP.west);

  \draw[cbd/arrow] (bdP.east) -- node[pos=0.5,above]{#3}
    coordinate[pos=0.5] (bdYmid)
    (bdY);
}

% Feedback (works in both envs):
% - With sum: returns to bdSum.south
% - No sum: returns to bdC.west (controller input)
\NewDocumentCommand{\feedback}{ O{} m }{%
  \coordinate (bdFBright) at ($(bdYmid)+(0,\pgfkeysvalueof{/blockdiagram/fbY})$);

  \ifbdHasSum
    \coordinate (bdFBleft) at ($(bdSum)+(0,\pgfkeysvalueof{/blockdiagram/fbY})$);
    \def\bdReturn{bdSum.south}
  \else
    \coordinate (bdFBleft) at ($(bdC.west)+(0,\pgfkeysvalueof{/blockdiagram/fbY})$);
    \def\bdReturn{bdC.west}
  \fi

  \IfEmptyTF{#1}{%
    \draw[cbd/arrow]
      (bdYmid) -- (bdFBright) --
      node[pos=0.35,above]{#2}
      (bdFBleft) -- (\bdReturn);
  }{%
    \coordinate (bdFBmid) at ($(bdFBright)!0.5!(bdFBleft)$);
    \node[cbd/block, minimum width=16mm, minimum height=10mm] (bdH) at (bdFBmid) {#1};

    \draw[cbd/arrow]
      (bdYmid) -- (bdFBright) --
      node[pos=0.18,above]{#2}
      (bdH.east);
    \draw[cbd/arrow] (bdH.west) -- (bdFBleft) -- (\bdReturn);
  }%
}

\NewDocumentCommand{\disturbance}{ O{+} O{1.6} m }{%
  % #1 = sign to display near disturbance input (+ or -)
  % #2 = vertical offset above the summer (positive number)
  % #3 = disturbance label
  %
  % Requires: bdSum node exists (i.e., use inside blockdiagram, not blockdiagramnosum)

  % disturbance source coordinate above the summer
  \coordinate (bdD) at ($(bdSum)+(0,#2)$);

  % draw disturbance arrow into the top of summer
  \draw[cbd/arrow] (bdD) -- node[pos=0.5,left]{#3} (bdSum.north);

  % mark sign near that input (just a small annotation near the north port)
  \node at ($(bdSum.north)+(0.25,-0.25)$) {\scriptsize $#1$};
}
%========================
% Examples
%========================
% WITH summing junction:
% \begin{blockdiagram}
%   \reference{$r(t)$}
%   \controller[$C(s)$]{$e(t)$}{$u(t)$}
%   \system[$P(s)$]{$u(t)$}{$y(t)$}
%   \feedback{$y(t)$}
% \end{blockdiagram}
%
% NO summing junction:
% \begin{blockdiagramnosum}
%   \reference{$r(t)$}
%   \controller[$C(s)$]{$r(t)$}{$u(t)$}
%   \system[$P(s)$]{$u(t)$}{$y(t)$}
% \end{blockdiagramnosum}
%
% Adjust spacing:
% \begin{blockdiagramnosum}[xC=3.8,xP=8.4,xY=12.0,fbY=-2.8]
%   ...
% \end{blockdiagramnosum}
%========================
